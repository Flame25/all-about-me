% Options for packages loaded elsewhere
% Options for packages loaded elsewhere
\PassOptionsToPackage{unicode}{hyperref}
\PassOptionsToPackage{hyphens}{url}
\PassOptionsToPackage{dvipsnames,svgnames,x11names}{xcolor}
%
\documentclass[
  letterpaper,
  DIV=11,
  numbers=noendperiod]{scrreprt}
\usepackage{xcolor}
\usepackage{amsmath,amssymb}
\setcounter{secnumdepth}{5}
\usepackage{iftex}
\ifPDFTeX
  \usepackage[T1]{fontenc}
  \usepackage[utf8]{inputenc}
  \usepackage{textcomp} % provide euro and other symbols
\else % if luatex or xetex
  \usepackage{unicode-math} % this also loads fontspec
  \defaultfontfeatures{Scale=MatchLowercase}
  \defaultfontfeatures[\rmfamily]{Ligatures=TeX,Scale=1}
\fi
\usepackage{lmodern}
\ifPDFTeX\else
  % xetex/luatex font selection
\fi
% Use upquote if available, for straight quotes in verbatim environments
\IfFileExists{upquote.sty}{\usepackage{upquote}}{}
\IfFileExists{microtype.sty}{% use microtype if available
  \usepackage[]{microtype}
  \UseMicrotypeSet[protrusion]{basicmath} % disable protrusion for tt fonts
}{}
\makeatletter
\@ifundefined{KOMAClassName}{% if non-KOMA class
  \IfFileExists{parskip.sty}{%
    \usepackage{parskip}
  }{% else
    \setlength{\parindent}{0pt}
    \setlength{\parskip}{6pt plus 2pt minus 1pt}}
}{% if KOMA class
  \KOMAoptions{parskip=half}}
\makeatother
% Make \paragraph and \subparagraph free-standing
\makeatletter
\ifx\paragraph\undefined\else
  \let\oldparagraph\paragraph
  \renewcommand{\paragraph}{
    \@ifstar
      \xxxParagraphStar
      \xxxParagraphNoStar
  }
  \newcommand{\xxxParagraphStar}[1]{\oldparagraph*{#1}\mbox{}}
  \newcommand{\xxxParagraphNoStar}[1]{\oldparagraph{#1}\mbox{}}
\fi
\ifx\subparagraph\undefined\else
  \let\oldsubparagraph\subparagraph
  \renewcommand{\subparagraph}{
    \@ifstar
      \xxxSubParagraphStar
      \xxxSubParagraphNoStar
  }
  \newcommand{\xxxSubParagraphStar}[1]{\oldsubparagraph*{#1}\mbox{}}
  \newcommand{\xxxSubParagraphNoStar}[1]{\oldsubparagraph{#1}\mbox{}}
\fi
\makeatother


\usepackage{longtable,booktabs,array}
\usepackage{calc} % for calculating minipage widths
% Correct order of tables after \paragraph or \subparagraph
\usepackage{etoolbox}
\makeatletter
\patchcmd\longtable{\par}{\if@noskipsec\mbox{}\fi\par}{}{}
\makeatother
% Allow footnotes in longtable head/foot
\IfFileExists{footnotehyper.sty}{\usepackage{footnotehyper}}{\usepackage{footnote}}
\makesavenoteenv{longtable}
\usepackage{graphicx}
\makeatletter
\newsavebox\pandoc@box
\newcommand*\pandocbounded[1]{% scales image to fit in text height/width
  \sbox\pandoc@box{#1}%
  \Gscale@div\@tempa{\textheight}{\dimexpr\ht\pandoc@box+\dp\pandoc@box\relax}%
  \Gscale@div\@tempb{\linewidth}{\wd\pandoc@box}%
  \ifdim\@tempb\p@<\@tempa\p@\let\@tempa\@tempb\fi% select the smaller of both
  \ifdim\@tempa\p@<\p@\scalebox{\@tempa}{\usebox\pandoc@box}%
  \else\usebox{\pandoc@box}%
  \fi%
}
% Set default figure placement to htbp
\def\fps@figure{htbp}
\makeatother





\setlength{\emergencystretch}{3em} % prevent overfull lines

\providecommand{\tightlist}{%
  \setlength{\itemsep}{0pt}\setlength{\parskip}{0pt}}



 


\KOMAoption{captions}{tableheading}
\makeatletter
\@ifpackageloaded{bookmark}{}{\usepackage{bookmark}}
\makeatother
\makeatletter
\@ifpackageloaded{caption}{}{\usepackage{caption}}
\AtBeginDocument{%
\ifdefined\contentsname
  \renewcommand*\contentsname{Table of contents}
\else
  \newcommand\contentsname{Table of contents}
\fi
\ifdefined\listfigurename
  \renewcommand*\listfigurename{List of Figures}
\else
  \newcommand\listfigurename{List of Figures}
\fi
\ifdefined\listtablename
  \renewcommand*\listtablename{List of Tables}
\else
  \newcommand\listtablename{List of Tables}
\fi
\ifdefined\figurename
  \renewcommand*\figurename{Figure}
\else
  \newcommand\figurename{Figure}
\fi
\ifdefined\tablename
  \renewcommand*\tablename{Table}
\else
  \newcommand\tablename{Table}
\fi
}
\@ifpackageloaded{float}{}{\usepackage{float}}
\floatstyle{ruled}
\@ifundefined{c@chapter}{\newfloat{codelisting}{h}{lop}}{\newfloat{codelisting}{h}{lop}[chapter]}
\floatname{codelisting}{Listing}
\newcommand*\listoflistings{\listof{codelisting}{List of Listings}}
\makeatother
\makeatletter
\makeatother
\makeatletter
\@ifpackageloaded{caption}{}{\usepackage{caption}}
\@ifpackageloaded{subcaption}{}{\usepackage{subcaption}}
\makeatother
\usepackage{bookmark}
\IfFileExists{xurl.sty}{\usepackage{xurl}}{} % add URL line breaks if available
\urlstyle{same}
\hypersetup{
  pdftitle={Christopher Richard C.},
  pdfauthor={18222057 Christopher Richard C.},
  colorlinks=true,
  linkcolor={blue},
  filecolor={Maroon},
  citecolor={Blue},
  urlcolor={Blue},
  pdfcreator={LaTeX via pandoc}}


\title{Christopher Richard C.}
\usepackage{etoolbox}
\makeatletter
\providecommand{\subtitle}[1]{% add subtitle to \maketitle
  \apptocmd{\@title}{\par {\large #1 \par}}{}{}
}
\makeatother
\subtitle{Portfolio Asesmen II-2100 KIPP}
\author{18222057 Christopher Richard C.}
\date{2025-10-20}
\begin{document}
\maketitle

\renewcommand*\contentsname{Table of contents}
{
\hypersetup{linkcolor=}
\setcounter{tocdepth}{2}
\tableofcontents
}

\bookmarksetup{startatroot}

\chapter*{Selamat Berjumpa}\label{selamat-berjumpa}
\addcontentsline{toc}{chapter}{Selamat Berjumpa}

\markboth{Selamat Berjumpa}{Selamat Berjumpa}

Halo! Saya \textbf{Christopher Richard Chandra}, senang sekali kamu
berkunjung ke repositori saya.\\
Di sini, saya mendokumentasikan perjalanan saya dalam mempelajari dan
mengasah keterampilan \textbf{komunikasi interpersonal} sebagai bagian
dari tugas mata kuliah \textbf{II2100}.

Di sini, saya mendokumentasikan perjalan saya dalam mempelajari dan
mengasah keterampilan Komunikasi interpersonal melalui teks, karya, dan
refleksi pribadi. Selama proses pembelajaran ini, saya banyak melakukan
eksplorasi melalui tulisan dan refleksi pribadi. Tujuannya adalah untuk
memahami diri saya sendiri sekaligus memperbaiki cara saya berkomunikasi
dengan orang lain.

Setiap konten yang ada di sini adalah hasil dari pemikiran singkat dan
eksploratif jadi mungkin tidak sepenuhnya mencerminkan diri saya secara
keseluruhan. Namun, saya berharap pembaca bisa merasakan perjalanan
belajar saya dan mendapatkan inspirasi darinya.

\bookmarksetup{startatroot}

\chapter{UTS-1 All About Me}\label{uts-1-all-about-me}

UAV yang dibuat oleh Tim Aksantara pada SAFMC 2025 (Dok. Aksantara)

Saya Christopher Richard Chandra, biasa dipanggil Christopher, merupakan
seorang mahasiswa Sistem dan Teknologi Informasi yang selalu tertarik
pada penerapan teori dan pengetahuan dasar menjadi produk akhir yang
berguna, serta bagaimana teori tersebut diterapkan dalam dunia nyata.
Rasa penasaran ini mendorong saya untuk terus belajar, bereksperimen,
dan mencoba mengubah ide menjadi sesuatu yang nyata melalui sebuah
proyek ataupun penelitian sederhana.

Oleh karena itu, saya sering memanfaatkan berbagai kesempatan yang ada
untuk terus mengembangkan diri, baik melalui lomba, proyek, maupun
riset. Terkadang, saya mengambil terlalu banyak peran sekaligus,
sehingga menyebabkan kelelahan dan beberapa hal lain menjadi tertunda.
Meskipun begitu, saya memiliki prinsip untuk memanfaatkan setiap
kesempatan sebaik mungkin, karena saya percaya bahwa pengalaman dan
pembelajaran yang didapat akan sangat berharga untuk pertumbuhan pribadi
dan profesional.

Namun, saya juga belajar bahwa terlalu banyak mengambil tanggung jawab
sekaligus dapat memengaruhi performa saya. Oleh karena itu, saya
berusaha untuk lebih bijak dalam mengatur prioritas, memastikan setiap
kegiatan yang saya ikuti bisa dijalankan dengan maksimal, sambil tetap
menjaga keseimbangan antara produktivitas dan kesehatan pribadi.

Salah satu keputusan terbaik yang pernah saya ambil adalah bergabung
dengan UKM Aksantara, di mana kami fokus pada pembuatan UAV (Unmanned
Aerial Vehicle) dan berpartisipasi dalam berbagai kompetisi, baik
nasional maupun internasional. UKM ini menjadi ruang saya untuk
berkembang dan eksplorasi mandiri dan mendapatkan sangat banyak
pengetahuan baru. Pada dasarnya ruang saya untuk berkarya seputar UAV,
seperti penerapan ilmu dasar yang dipelajari pada saat perkuliahan
hingga hal - hal \emph{advance} yang saya peroleh pada saat eksplorasi.
Melalui pengalaman ini, saya belajar banyak hal baru, terekspos pada
peluang-peluang menarik, serta menemukan teknologi dan metode yang
sebelumnya belum pernah saya ketahui.

Selain itu, saya memiliki kesempatan untuk bekerja bersama teman-teman
yang memiliki minat sama, yang membuat proses belajar menjadi lebih
menyenangkan dan memotivasi. Bergabung dengan UKM ini tidak hanya
memperluas pengetahuan teknis saya, tetapi juga membantu saya
mengembangkan keterampilan kerja sama, manajemen proyek, dan kemampuan
beradaptasi dalam lingkungan yang kompetitif.

Pengalaman singkat saya selama di ITB yang selama ini saya lalui
meyakini saya bahwa bahwa setiap pengalaman, baik berhasil maupun gagal
adalah kesempatan untuk belajar dan berkembang. Saya dan sangat
berharga. Tidak ada orang yang bisa langsung sukses pasti ada gagal nya
dahulu baik itu kegagalan yang besar maupun kecil dan selalu berusaha
memanfaatkan setiap kesempatan yang datang karena kita tidak pernah tahu
pengalaman apa yang bisa membuka pintu baru.

\bookmarksetup{startatroot}

\chapter{UTS-2 My Songs for You}\label{uts-2-my-songs-for-you}

\section{Lanterns In a Dark}\label{lanterns-in-a-dark}

\textbf{Lanterns In a Dark} adalah refleksi pribadi saya tentang
perjalanan belajar, pertumbuhan, dan prinsip hidup yang saya pegang.
Setiap bagian lirik mencerminkan pengalaman dan filosofi saya dalam
menghadapi tantangan, belajar hal baru, dan mengembangkan diri.

Your browser does not support the audio element.

\begin{center}\rule{0.5\linewidth}{0.5pt}\end{center}

\textbf{Verse 1}

\begin{quote}
I wander through the quiet spaces,\\
Where thought meets sky, and curiosity traces\\
The hidden patterns in the world around,\\
Listening to whispers, where answers are found.
\end{quote}

\textbf{Pre-Chorus}

\begin{quote}
I may stumble, I may fall,\\
Yet I rise, I heed the call.
\end{quote}

\textbf{Chorus}

\begin{quote}
I carry my principles like lanterns in the dark,\\
Guiding my journey, igniting a spark.\\
Through shadow and storm, through calm and fight,\\
I shape my path, I reach for the light.
\end{quote}

\textbf{Verse 2}

\begin{quote}
I take on challenges, both small and grand,\\
Turning thoughts to action with careful hands,\\
Each step a lesson, each challenge a spark,\\
Illuminating corners that were once dark.
\end{quote}

\textbf{Bridge}

\begin{quote}
It's not about being loud or first to shine,\\
But finding meaning, one choice at a time,\\
In every stumble, in every try,\\
I discover the strength I carry inside.
\end{quote}

\textbf{Outro}

\begin{quote}
I walk my path, both steady and free,\\
A mind exploring, a soul that sees,\\
In every breath, in all I pursue,\\
I live my principles, and remain true.
\end{quote}

\bookmarksetup{startatroot}

\chapter{UTS-3 My Stories for You}\label{uts-3-my-stories-for-you}

\section{Perjalanan Saya di Aksantara
ITB}\label{perjalanan-saya-di-aksantara-itb}

Saya masuk Aksantara ITB pada semester 4 dan bagi saya ini adalah salah
satu keputusan terbaik yang pernah saya ambil. Dari awal, pengalaman ini
terasa berbeda dari UKM pada umumnya. Tidak seperti beberapa UKM atau
himpunan lain yang menekankan formalitas ospek, Aksantara lebih fokus
pada praktik dan pembelajaran langsung.

Tahap awal yang harus kami lalui adalah pendidikan dasar dimana kami
belajar tentang pembuatan UAV (drone) dan pemrogramannya. Saya
ditempatkan di tim VTOL yang fokus pada pembuatan drone dengan misi
khusus. Awalnya, banyak hal terasa membingungkan, mulai dari desain
drone hingga kode-kode pemrograman untuk membuat drone bergerak
otomatis. Namun, seiring berjalannya waktu, hal ini mulai menjadi sangat
menarik karena melihat drone bergerak sendiri tanpa intervensi manusia
adalah pengalaman yang luar biasa.

Pengalaman ini juga menjadi entry level bagi saya untuk terjun lebih
dalam ke dunia UAV. Saya sangat bersyukur kepada ketua tim sebelumnya
yang mau benar-benar mengajarkan kami, membimbing, dan membantu kami
merasakan sense of belonging terhadap tim ini sehingga kami benar-benar
akrab dan bekerja sebagai satu tim. Kami hampir bekerja setiap hari dan
tidak terasa tiga bulan berlalu sampai persiapan menuju kompetisi di
Jogja.

Di Jogja, saya merasa masih sangat tidak siap. Banyak fitur belum
selesai dan pengujian belum sepenuhnya rampung. Kami terus melakukan
pengujian dari pagi hingga pagi berikutnya. Hasilnya, kami kalah di 8
besar. Semua anggota tim sedih karena merasa usaha kami sia-sia, tapi
bagi saya, pengalaman ini sangat berharga. Saya belajar banyak hal
teknis maupun non-teknis sehingga kekalahan itu tidak terasa
menyedihkan.

Beberapa bulan setelah itu, saya mengajukan diri dan terpilih mewakili
Aksantara untuk lomba internasional di Singapura yang memiliki tingkat
kompleksitas jauh lebih tinggi karena semua harus dilakukan indoor.
Tantangan ini menjadi kesempatan besar untuk mengembangkan kemampuan
saya lebih jauh dan pengalaman sebelumnya membuat saya lebih siap
menghadapi situasi yang kompleks dan penuh tekanan.

\section{Belajar dari Tick, Tick\ldots{}
Boom!}\label{belajar-dari-tick-tick-boom}

Beberapa waktu lalu, saya menonton film ``Tick, Tick\ldots{} Boom!''
yang menceritakan perjalanan Jonathan Larson, seorang penulis dan
komponis muda yang berjuang mengejar impian di dunia teater. Film ini
unik karena bukan hanya tentang kehidupannya secara umum, tetapi secara
khusus menceritakan proses Larson membuat musikal ``Tick, Tick\ldots{}
Boom!'' itu sendiri. Kita bisa melihat dengan jelas bagaimana ide,
kegelisahan, dan pengalaman pribadinya dituangkan ke dalam karya yang ia
cintai.

Di film ini, Larson digambarkan sebagai sosok yang penuh semangat,
kreatif, dan konyol. Kita bisa melihat bagaimana ia menjalani
hari-harinya: menulis musik, berdiskusi dengan teman, menghadapi tekanan
pekerjaan, dan mengatur waktunya untuk menyelesaikan musikalnya. Banyak
momen lucu dan menghibur, misalnya ketika ia frustrasi dengan hal-hal
kecil atau ketika ide-idenya muncul secara tiba-tiba dan konyol, tetapi
semuanya mencerminkan dedikasinya untuk berkarya.

Salah satu hal paling menyentuh adalah musikalnya sendiri --- ``Tick,
Tick\ldots{} Boom!'' --- yang ramai dan penuh antusiasme. Larson bekerja
keras untuk membuat karya ini, dan orang-orang di sekitarnya
meresponsnya dengan penuh semangat. Sayangnya, tragisnya Larson
meninggal sebelum sempat melihat kesuksesan penuh dari karyanya sendiri,
meninggalkan cerita tentang seorang kreator yang gigih mengejar passion
hingga akhir hayatnya.

Menonton film ini membuat saya merenung tentang perjalanan pribadi saya
sendiri. Seperti Larson, saya juga menghadapi ketidakpastian dan
tantangan dalam mengejar hal-hal yang saya minati, mulai dari belajar
hal baru hingga menghadapi kegagalan. Saya belajar bahwa ketekunan,
keberanian, dan konsistensi jauh lebih penting daripada hasil instan.

Film ini juga mengajarkan saya tentang pentingnya teman dan komunitas.
Larson dikelilingi teman-teman yang mendukung, mengkritik, dan
menantangnya untuk menjadi lebih baik. Lingkungan yang positif sangat
membantu kita untuk tetap termotivasi dan berkembang, terutama ketika
menghadapi tantangan sulit.

Dari ``Tick, Tick\ldots{} Boom!'', saya belajar bahwa waktu terus
berjalan --- tick, tick, boom --- dan setiap detik bisa diisi dengan
keberanian, kreativitas, dan usaha nyata. Cerita Larson mengingatkan
saya untuk tidak takut mencoba, terus belajar, dan mengejar apa yang
benar-benar saya inginkan, meskipun jalan di depan penuh rintangan dan
ketidakpastian.

\bookmarksetup{startatroot}

\chapter{UTS-4 My SHAPE (Spiritual Gifts, Heart, Abilities, Personality,
Experiences)}\label{uts-4-my-shape-spiritual-gifts-heart-abilities-personality-experiences}

\begin{quote}
\textbf{Tujuan:} Merangkum rancangan diri (charter) agar saya melayani,
berkarya, dan memimpin secara paling selaras dengan karunia dan
pengalaman hidup saya. Dapat langsung ditempel ke halaman \textbf{UTS-4
--- My SHAPE} dan dipakai sebagai acuan aksi 90 hari.
\end{quote}

\section{\texorpdfstring{Sumber
\href{StrengthsProfile-Christopher-Richard.pdf}{VIA
assessment}}{Sumber VIA assessment}}\label{sumber-via-assessment}

\section{Piagam Diri}\label{piagam-diri}

\begin{longtable}[]{@{}
  >{\raggedright\arraybackslash}p{(\linewidth - 4\tabcolsep) * \real{0.3333}}
  >{\raggedright\arraybackslash}p{(\linewidth - 4\tabcolsep) * \real{0.3333}}
  >{\raggedright\arraybackslash}p{(\linewidth - 4\tabcolsep) * \real{0.3333}}@{}}
\toprule\noalign{}
\begin{minipage}[b]{\linewidth}\raggedright
Kategori
\end{minipage} & \begin{minipage}[b]{\linewidth}\raggedright
Deskripsi
\end{minipage} & \begin{minipage}[b]{\linewidth}\raggedright
\end{minipage} \\
\midrule\noalign{}
\endhead
\bottomrule\noalign{}
\endlastfoot
\textbf{Signature Strengths (Kekuatan Khas)} & Apabila mengutip
berdasarkan hasil \emph{VIA Assesment}, saya adalah individu yang
cenderung melihat sisi terang dan menikmati proses membawa senyuman
kepada orang lain. Saya bekerja secara efektif sebagai bagian dari tim,
menunjukkan loyalitas kepada kelompok, dan selalu memastikan saya
melakukan bagian saya. Saya memiliki bakat alami untuk memikirkan
cara-cara yang baru dan produktif untuk mengkonseptualisasikan dan
melakukan berbagai hal. Dorongan ini dilengkapi dengan antusiasme yang
kuat untuk menguasai keterampilan dan pengetahuan baru, baik secara
mandiri maupun formal, dengan kecenderungan untuk terus menambah apa
yang saya ketahui secara sistematis. & \\
\textbf{Heart (Nilai Inti \& Motivasi Intrinsik)} & Saya sangat
menghargai integritas dalam hidup saya, karena prinsip ini menjadi
fondasi dalam setiap keputusan dan interaksi yang saya lakukan. Selain
itu, saya terdorong oleh semangat petualangan dan eksplorasi, selalu
ingin mencoba hal-hal baru dan memahami berbagai sudut pandang. Minat
saya terhadap pengembangan ilmu pengetahuan dan pemecahan masalah
membuat saya senang berdiskusi panjang lebar tentang ide-ide baru,
inovasi, atau tantangan kompleks yang menuntut logika dan kreativitas.
Nilai dan minat ini saling melengkapi: integritas memastikan saya tetap
konsisten dengan prinsip, sementara rasa ingin tahu dan semangat
eksplorasi mendorong saya untuk terus belajar, tumbuh, dan menghadapi
tantangan dengan antusiasme. & \\
\textbf{Aptitudes \& Acquired Skills (Bakat \& Keterampilan)} & Selama
beberapa proyek dan pengalaman organisasi yang saya jalani, saya
mengembangkan berbagai keterampilan teknis seperti pemrograman, analisis
data, simulasi, desain, dan dokumentasi teknis yang rapi. Selain itu,
saya juga memperkuat soft skills, termasuk kepemimpinan tim, komunikasi
efektif dengan rekan dan mentor, manajemen waktu menghadapi deadline,
adaptabilitas saat menghadapi perubahan, serta kemampuan memecahkan
masalah secara kreatif. Semua keterampilan ini diuji dan diterapkan
dalam berbagai pencapaian nyata, mulai dari proyek kelompok, kompetisi
hingga pengalaman representatif di tingkat internasional yang memberi
saya wawasan berharga dalam bekerja di lingkungan kolaboratif dan
kompetitif. & \\
\textbf{Personality (Kepribadian)} & Saya merupakan seorang INTP, yang
cenderung berpikir secara analitis, kritis, dan suka mengeksplorasi
berbagai ide. Saya senang merenung, memecahkan masalah, dan memahami
konsep secara mendalam sebelum mengambil kesimpulan. Kadang, saya
terlalu tenggelam dalam pemikiran saya sendiri sehingga perlu belajar
untuk mengekspresikan ide secara lebih terbuka dan berkolaborasi dengan
orang lain. Meski demikian, saya selalu berusaha menjaga fleksibilitas
dan terbuka terhadap perspektif baru karena saya percaya bahwa ide besar
sering lahir dari interaksi, diskusi, dan inspirasi dari orang lain.
& \\
\textbf{Experiences (Pengalaman Hidup)} & Sepanjang hidup, saya telah
melalui banyak pengalaman dan peristiwa yang membentuk siapa saya
sekarang. Namun, yang paling berkesan bagi saya adalah momen-momen
kegagalan karena dari situ saya belajar lebih banyak daripada dari
keberhasilan. Setiap kegagalan saya jadikan kesempatan untuk merenung,
menganalisis alasan kegagalan, dan memastikan saya tidak mengulang
kesalahan yang sama. Filosofi ini menjadi pedoman hidup saya: menghadapi
tantangan dengan kepala dingin, terus berusaha, dan tidak menyerah meski
hasil tidak sesuai harapan. Kegagalan bukan akhir, melainkan guru yang
mengajarkan cara menjadi lebih baik, lebih bijak, dan lebih tangguh
dalam menghadapi segala hal di masa depan. & \\
\end{longtable}

\section{Pernyataan Misi Pribadi}\label{pernyataan-misi-pribadi}

\begin{quote}
Saya ingin ilmu yang saya pelajari bisa diterapkan untuk hal-hal
positif. Karena itu, saya terus belajar dan mengasah keterampilan.
Setiap langkah yang saya ambil diharapkan memberi dampak nyata bagi diri
sendiri dan orang lain.
\end{quote}

\section{Identitas Naratif}\label{identitas-naratif}

Sejak kecil, saya selalu tertarik pada ide-ide baru dan cara-cara untuk
memahami dunia di sekitar saya. Saya senang merenung, mengeksplorasi
konsep, dan memecahkan masalah secara logis. Sifat analitis ini membuat
saya sering tenggelam dalam pemikiran sendiri, tetapi juga mendorong
saya untuk terus mencari jawaban yang lebih dalam dan kreatif.

Nilai yang saya pegang kuat adalah integritas dan semangat eksplorasi.
Integritas membuat saya selalu berusaha konsisten dengan prinsip dan
keputusan saya, sementara semangat eksplorasi mendorong saya untuk
mencoba hal-hal baru, belajar dari pengalaman, dan terbuka terhadap
perspektif lain. Kedua nilai ini membentuk cara saya berinteraksi dengan
orang lain dan menghadapi tantangan.

Sepanjang perjalanan hidup, saya belajar banyak dari kegagalan.
Momen-momen ketika rencana saya tidak berjalan sesuai harapan
mengajarkan saya untuk merenung, memahami kesalahan, dan memastikan saya
tumbuh dari pengalaman tersebut. Kegagalan bagi saya bukan akhir, tetapi
guru yang memberi kesempatan untuk menjadi lebih bijak, kreatif, dan
tangguh.

Misi pribadi saya adalah menciptakan sesuatu yang bermanfaat bagi orang
banyak. Saya ingin ilmu yang saya pelajari dan keterampilan yang saya
asah bisa diterapkan untuk hal-hal positif. Untuk itu, saya terus
belajar, mengeksplorasi ide, dan mencari cara agar setiap langkah yang
saya ambil memberi dampak nyata bagi diri sendiri maupun orang lain.

\bookmarksetup{startatroot}

\chapter{UTS-5 My Personal Reviews}\label{uts-5-my-personal-reviews}

Berikut cara saya melakukan review: mengguan chatGPT, saya mengattach
\href{skor_uts.pdf}{file promt ChatGPT}, disertai perintah :``self
assess uts-1 sanpai uts-5 dari URL
`https://ii-2100.github.io/all-about-me/'\,''

ChatGPT melakukan self-assessment UTS-1 s.d. UTS-5 langsung dari laman
yang Anda berikan dan menilai memakai rubrik tugas UTS (skala 1--5 per
kriteria). Rekap skor siap diunduh sebagai CSV:
\href{sandbox:/mnt/data/UTS_self_assessment.csv}{Download CSV
ringkasan}.

\bookmarksetup{startatroot}

\chapter{Hasil Self-Assessment UTS (URL:
ii-2100.github.io/all-about-me)}\label{hasil-self-assessment-uts-url-ii-2100.github.ioall-about-me}

\section{Identifikasi}\label{identifikasi}

\begin{itemize}
\tightlist
\item
  Nama \& NIM penulis: \textbf{Armein Z. R. Langi -- 131902360} (tertera
  di halaman depan portofolio).
  (\href{https://ii-2100.github.io/all-about-me/}{II 2100})
\item
  Penilai: \textbf{Self-assessment (Armein Z. R. Langi)}
\item
  Catatan cakupan: halaman beranda memuat ``About Me''; navigasi ke ``My
  Songs for You'', ``My Stories for You'', ``My Shapes'', dan ``My
  Personal Reviews'' tersedia.
  (\href{https://ii-2100.github.io/all-about-me/}{II 2100})
\end{itemize}

\section{Tinjauan Umum}\label{tinjauan-umum}

\begin{itemize}
\tightlist
\item
  \textbf{UTS-1 (All About Me)} hadir di beranda (``Selamat Berjumpa /
  About Me''). Isi memperkenalkan identitas dan latar personal secara
  padat. (\href{https://ii-2100.github.io/all-about-me/}{II 2100})
\item
  \textbf{UTS-2 (My Songs for You)} memuat judul karya dan tautan audio,
  namun lirik/isi tidak ditampilkan di halaman (file audio tidak bisa
  saya akses dari sini), sehingga penilaian konten terbatas pada
  kelengkapan presentasi.
  (\href{https://ii-2100.github.io/all-about-me/My_Song_for_You/index.html}{II
  2100})
\item
  \textbf{UTS-3 (My Stories for You)} berisi tautan ke beberapa cerita;
  konten naratif kuat, terutama pada kisah keluarga ``Gaun Pengantin
  Gladys'' serta kisah inspiratif lainnya.
  (\href{https://ii-2100.github.io/all-about-me/My_Stories_for_You/index.html}{II
  2100})
\item
  \textbf{UTS-4 (My SHAPE)} saat ini halaman kosong/tanpa isi, hanya
  navigasi.
  (\href{https://ii-2100.github.io/all-about-me/My_Shapes/index.html}{II
  2100})
\item
  \textbf{UTS-5 (My Personal Reviews)} berisi metode/tautan panduan
  review, namun belum ada contoh \textbf{review personal} yang lengkap
  terhadap sebuah pesan/teks.
  (\href{https://ii-2100.github.io/all-about-me/My_Personal_Reviews/index.html}{II
  2100})
\end{itemize}

\begin{center}\rule{0.5\linewidth}{0.5pt}\end{center}

\section{Tinjauan Spesifik + Skor
(1--5)}\label{tinjauan-spesifik-skor-15}

\subsection{UTS-1 --- All About Me (di
beranda)}\label{uts-1-all-about-me-di-beranda}

\textbf{Skor per kriteria:} Orisinalitas \textbf{3}, Keterlibatan
\textbf{3}, Humor \textbf{1}, Wawasan/Insight \textbf{3} → \textbf{Total
10/20 (50\%)}. \textbf{Alasan singkat:} Perkenalan diri faktual dan
jelas, namun gaya masih informatif ketimbang naratif (minim humor dan
moment ``hook''). (\href{https://ii-2100.github.io/all-about-me/}{II
2100}) \textbf{Saran perbaikan:} Buka dengan anekdot pribadi (1--2
paragraf) yang ``mengikat'' (mis. titik balik karier/keluarga),
tambahkan satu momen humor ringan, lalu tutup dengan refleksi singkat
tentang nilai/visi diri agar aspek insight naik.

\subsection{UTS-2 --- My Songs for You}\label{uts-2-my-songs-for-you-1}

\textbf{Skor per kriteria:} Orisinalitas \textbf{2}, Keterlibatan
\textbf{2}, Humor \textbf{1}, Inspirasi \textbf{2} → \textbf{Total 7/20
(35\%)}. \textbf{Alasan singkat:} Halaman menampilkan judul lagu ``River
in my Mind'', ``Heaven on Earth'', namun tanpa lirik/cerita di balik
lagu sehingga sulit menilai aspek pesan, humor, dan inspirasi.
(\href{https://ii-2100.github.io/all-about-me/My_Song_for_You/index.html}{II
2100}) \textbf{Saran perbaikan:} Tambahkan lirik lengkap, 1 paragraf
cerita proses kreatif, dan 2--3 kalimat ``pesan untukmu'' agar inspirasi
terbaca; sertakan player/tautan yang dapat diputar langsung + fallback
transkrip.

\subsection{UTS-3 --- My Stories for
You}\label{uts-3-my-stories-for-you-1}

\textbf{Skor per kriteria:} Orisinalitas \textbf{5}, Keterlibatan
\textbf{5}, Pengembangan Narasi \textbf{4}, Inspirasi \textbf{5} →
\textbf{Total 19/20 (95\%)}. \textbf{Alasan singkat:} Cerita ``Gaun
Pengantin Gladys'' dkk sangat personal, emosional, dan inspiratif; ritme
bertutur hidup serta detail situasional kuat (konten ramu unsur
penebusan/keluarga/iman).
(\href{https://ii-2100.github.io/all-about-me/My_Stories_for_You/index.html}{II
2100}) \textbf{Saran perbaikan:} Tambah ``lead'' 2--3 kalimat yang
merangkum pesan kunci tiap cerita; akhiri dengan ajakan/refleksi 1--2
kalimat agar resonansi ke pembaca makin jelas.

\subsection{UTS-4 --- My SHAPE}\label{uts-4-my-shape}

\textbf{Skor per kriteria:} Orisinalitas \textbf{1}, Keterlibatan
\textbf{1}, Pengembangan \textbf{1}, Inspirasi \textbf{1} →
\textbf{Total 4/20 (20\%)}. \textbf{Alasan singkat:} Konten belum
tersedia.
(\href{https://ii-2100.github.io/all-about-me/My_Shapes/index.html}{II
2100}) \textbf{Saran perbaikan (prioritas):}

\begin{enumerate}
\def\labelenumi{\arabic{enumi}.}
\tightlist
\item
  Tampilkan rekap SHAPE (Spiritual gifts, Heart, Abilities, Personality,
  Experiences) dalam 1 tabel ringkas + 3--5 butir bukti;
\item
  Narasikan 2 pengalaman kunci (±150 kata/each) yang menunjukkan
  ``signature strengths'';
\item
  Tutup dengan 3 rencana aksi konkret (who/what/when).
\end{enumerate}

\subsection{UTS-5 --- My Personal
Reviews}\label{uts-5-my-personal-reviews-1}

\textbf{Skor per kriteria:} Pemahaman Konsep \textbf{2}, Analisis Kritis
\textbf{1}, Argumentasi (Logos) \textbf{1}, Etos \& Empati \textbf{2},
Rekomendasi \textbf{1} → \textbf{Total 7/25 (28\%)}. \textbf{Alasan
singkat:} Halaman berisi metode cara menilai, tetapi \textbf{belum ada}
contoh \textbf{review personal} yang lengkap terhadap sebuah pesan/teks
sehingga aspek analisis-argumentasi tak terbaca.
(\href{https://ii-2100.github.io/all-about-me/My_Personal_Reviews/index.html}{II
2100}) \textbf{Saran perbaikan:} Pilih 1 karya personal (mis.
UTS-1/2/3), tulis review 400--600 kata: ringkas pesan, nilai dengan
rubrik, berikan 2--3 bukti kutipan, evaluasi etos/empati, lalu tutup
dengan rekomendasi perbaikan spesifik.

\begin{center}\rule{0.5\linewidth}{0.5pt}\end{center}

\section{Rekap Skor (ringkas)}\label{rekap-skor-ringkas}

\begin{itemize}
\tightlist
\item
  \textbf{UTS-1:} 10/20 → \textbf{50\%}
\item
  \textbf{UTS-2:} 7/20 → \textbf{35\%}
\item
  \textbf{UTS-3:} 19/20 → \textbf{95\%}
\item
  \textbf{UTS-4:} 4/20 → \textbf{20\%}
\item
  \textbf{UTS-5:} 7/25 → \textbf{28\%}
\end{itemize}

CSV lengkap sudah saya siapkan untuk dokumentasi dan olah lanjut:
\href{sandbox:/mnt/data/UTS_self_assessment.csv}{Download CSV
ringkasan}.

\section{Langkah Perbaikan Cepat (prioritas 1
minggu)}\label{langkah-perbaikan-cepat-prioritas-1-minggu}

\begin{enumerate}
\def\labelenumi{\arabic{enumi}.}
\tightlist
\item
  \textbf{Lengkapi UTS-4 (My SHAPE)} sesuai saran di atas;
\item
  \textbf{Tambahkan lirik + sinopsis} untuk tiap lagu di UTS-2 agar bisa
  dinilai isi/inspirasinya;
\item
  \textbf{Tulis 1 review personal penuh} (UTS-5) memakai rubrik (pilih
  salah satu karya Anda sendiri sebagai objek);
\item
  Revisi UTS-1 dengan pembuka anekdot + refleksi nilai diri.
\end{enumerate}

Jika Anda mau, saya bisa bantu merapikan UTS-4 (tabel SHAPE + narasi)
dan membuat kerangka cepat untuk \textbf{review} di UTS-5 dari salah
satu karya Anda.

\bookmarksetup{startatroot}

\chapter{Summary}\label{summary}

In summary, this book has no content whatsoever.

\bookmarksetup{startatroot}

\chapter*{References}\label{references}
\addcontentsline{toc}{chapter}{References}

\markboth{References}{References}

\phantomsection\label{refs}




\end{document}
